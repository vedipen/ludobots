% Generated by Sphinx.
\def\sphinxdocclass{report}
\newif\ifsphinxKeepOldNames \sphinxKeepOldNamestrue
\documentclass[letterpaper,10pt,english]{sphinxmanual}
\usepackage{iftex}

\ifPDFTeX
  \usepackage[utf8]{inputenc}
\fi
\ifdefined\DeclareUnicodeCharacter
  \DeclareUnicodeCharacter{00A0}{\nobreakspace}
\fi
\usepackage{cmap}
\usepackage[T1]{fontenc}
\usepackage{amsmath,amssymb,amstext}
\usepackage{babel}
\usepackage{times}
\usepackage[Bjarne]{fncychap}
\usepackage{longtable}
\usepackage{sphinx}
\usepackage{multirow}
\usepackage{eqparbox}


\addto\captionsenglish{\renewcommand{\figurename}{Fig.\@ }}
\addto\captionsenglish{\renewcommand{\tablename}{Table }}
\SetupFloatingEnvironment{literal-block}{name=Listing }

\addto\extrasenglish{\def\pageautorefname{page}}

\setcounter{tocdepth}{1}


\title{pyrosim Documentation}
\date{Jun 05, 2017}
\release{0.0.1}
\author{Collin Cappelle}
\newcommand{\sphinxlogo}{}
\renewcommand{\releasename}{Release}
\makeindex

\makeatletter
\def\PYG@reset{\let\PYG@it=\relax \let\PYG@bf=\relax%
    \let\PYG@ul=\relax \let\PYG@tc=\relax%
    \let\PYG@bc=\relax \let\PYG@ff=\relax}
\def\PYG@tok#1{\csname PYG@tok@#1\endcsname}
\def\PYG@toks#1+{\ifx\relax#1\empty\else%
    \PYG@tok{#1}\expandafter\PYG@toks\fi}
\def\PYG@do#1{\PYG@bc{\PYG@tc{\PYG@ul{%
    \PYG@it{\PYG@bf{\PYG@ff{#1}}}}}}}
\def\PYG#1#2{\PYG@reset\PYG@toks#1+\relax+\PYG@do{#2}}

\expandafter\def\csname PYG@tok@gd\endcsname{\def\PYG@tc##1{\textcolor[rgb]{0.63,0.00,0.00}{##1}}}
\expandafter\def\csname PYG@tok@gu\endcsname{\let\PYG@bf=\textbf\def\PYG@tc##1{\textcolor[rgb]{0.50,0.00,0.50}{##1}}}
\expandafter\def\csname PYG@tok@gt\endcsname{\def\PYG@tc##1{\textcolor[rgb]{0.00,0.27,0.87}{##1}}}
\expandafter\def\csname PYG@tok@gs\endcsname{\let\PYG@bf=\textbf}
\expandafter\def\csname PYG@tok@gr\endcsname{\def\PYG@tc##1{\textcolor[rgb]{1.00,0.00,0.00}{##1}}}
\expandafter\def\csname PYG@tok@cm\endcsname{\let\PYG@it=\textit\def\PYG@tc##1{\textcolor[rgb]{0.25,0.50,0.56}{##1}}}
\expandafter\def\csname PYG@tok@vg\endcsname{\def\PYG@tc##1{\textcolor[rgb]{0.73,0.38,0.84}{##1}}}
\expandafter\def\csname PYG@tok@vi\endcsname{\def\PYG@tc##1{\textcolor[rgb]{0.73,0.38,0.84}{##1}}}
\expandafter\def\csname PYG@tok@mh\endcsname{\def\PYG@tc##1{\textcolor[rgb]{0.13,0.50,0.31}{##1}}}
\expandafter\def\csname PYG@tok@cs\endcsname{\def\PYG@tc##1{\textcolor[rgb]{0.25,0.50,0.56}{##1}}\def\PYG@bc##1{\setlength{\fboxsep}{0pt}\colorbox[rgb]{1.00,0.94,0.94}{\strut ##1}}}
\expandafter\def\csname PYG@tok@ge\endcsname{\let\PYG@it=\textit}
\expandafter\def\csname PYG@tok@vc\endcsname{\def\PYG@tc##1{\textcolor[rgb]{0.73,0.38,0.84}{##1}}}
\expandafter\def\csname PYG@tok@il\endcsname{\def\PYG@tc##1{\textcolor[rgb]{0.13,0.50,0.31}{##1}}}
\expandafter\def\csname PYG@tok@go\endcsname{\def\PYG@tc##1{\textcolor[rgb]{0.20,0.20,0.20}{##1}}}
\expandafter\def\csname PYG@tok@cp\endcsname{\def\PYG@tc##1{\textcolor[rgb]{0.00,0.44,0.13}{##1}}}
\expandafter\def\csname PYG@tok@gi\endcsname{\def\PYG@tc##1{\textcolor[rgb]{0.00,0.63,0.00}{##1}}}
\expandafter\def\csname PYG@tok@gh\endcsname{\let\PYG@bf=\textbf\def\PYG@tc##1{\textcolor[rgb]{0.00,0.00,0.50}{##1}}}
\expandafter\def\csname PYG@tok@ni\endcsname{\let\PYG@bf=\textbf\def\PYG@tc##1{\textcolor[rgb]{0.84,0.33,0.22}{##1}}}
\expandafter\def\csname PYG@tok@nl\endcsname{\let\PYG@bf=\textbf\def\PYG@tc##1{\textcolor[rgb]{0.00,0.13,0.44}{##1}}}
\expandafter\def\csname PYG@tok@nn\endcsname{\let\PYG@bf=\textbf\def\PYG@tc##1{\textcolor[rgb]{0.05,0.52,0.71}{##1}}}
\expandafter\def\csname PYG@tok@no\endcsname{\def\PYG@tc##1{\textcolor[rgb]{0.38,0.68,0.84}{##1}}}
\expandafter\def\csname PYG@tok@na\endcsname{\def\PYG@tc##1{\textcolor[rgb]{0.25,0.44,0.63}{##1}}}
\expandafter\def\csname PYG@tok@nb\endcsname{\def\PYG@tc##1{\textcolor[rgb]{0.00,0.44,0.13}{##1}}}
\expandafter\def\csname PYG@tok@nc\endcsname{\let\PYG@bf=\textbf\def\PYG@tc##1{\textcolor[rgb]{0.05,0.52,0.71}{##1}}}
\expandafter\def\csname PYG@tok@nd\endcsname{\let\PYG@bf=\textbf\def\PYG@tc##1{\textcolor[rgb]{0.33,0.33,0.33}{##1}}}
\expandafter\def\csname PYG@tok@ne\endcsname{\def\PYG@tc##1{\textcolor[rgb]{0.00,0.44,0.13}{##1}}}
\expandafter\def\csname PYG@tok@nf\endcsname{\def\PYG@tc##1{\textcolor[rgb]{0.02,0.16,0.49}{##1}}}
\expandafter\def\csname PYG@tok@si\endcsname{\let\PYG@it=\textit\def\PYG@tc##1{\textcolor[rgb]{0.44,0.63,0.82}{##1}}}
\expandafter\def\csname PYG@tok@s2\endcsname{\def\PYG@tc##1{\textcolor[rgb]{0.25,0.44,0.63}{##1}}}
\expandafter\def\csname PYG@tok@nt\endcsname{\let\PYG@bf=\textbf\def\PYG@tc##1{\textcolor[rgb]{0.02,0.16,0.45}{##1}}}
\expandafter\def\csname PYG@tok@nv\endcsname{\def\PYG@tc##1{\textcolor[rgb]{0.73,0.38,0.84}{##1}}}
\expandafter\def\csname PYG@tok@s1\endcsname{\def\PYG@tc##1{\textcolor[rgb]{0.25,0.44,0.63}{##1}}}
\expandafter\def\csname PYG@tok@ch\endcsname{\let\PYG@it=\textit\def\PYG@tc##1{\textcolor[rgb]{0.25,0.50,0.56}{##1}}}
\expandafter\def\csname PYG@tok@m\endcsname{\def\PYG@tc##1{\textcolor[rgb]{0.13,0.50,0.31}{##1}}}
\expandafter\def\csname PYG@tok@gp\endcsname{\let\PYG@bf=\textbf\def\PYG@tc##1{\textcolor[rgb]{0.78,0.36,0.04}{##1}}}
\expandafter\def\csname PYG@tok@sh\endcsname{\def\PYG@tc##1{\textcolor[rgb]{0.25,0.44,0.63}{##1}}}
\expandafter\def\csname PYG@tok@ow\endcsname{\let\PYG@bf=\textbf\def\PYG@tc##1{\textcolor[rgb]{0.00,0.44,0.13}{##1}}}
\expandafter\def\csname PYG@tok@sx\endcsname{\def\PYG@tc##1{\textcolor[rgb]{0.78,0.36,0.04}{##1}}}
\expandafter\def\csname PYG@tok@bp\endcsname{\def\PYG@tc##1{\textcolor[rgb]{0.00,0.44,0.13}{##1}}}
\expandafter\def\csname PYG@tok@c1\endcsname{\let\PYG@it=\textit\def\PYG@tc##1{\textcolor[rgb]{0.25,0.50,0.56}{##1}}}
\expandafter\def\csname PYG@tok@o\endcsname{\def\PYG@tc##1{\textcolor[rgb]{0.40,0.40,0.40}{##1}}}
\expandafter\def\csname PYG@tok@kc\endcsname{\let\PYG@bf=\textbf\def\PYG@tc##1{\textcolor[rgb]{0.00,0.44,0.13}{##1}}}
\expandafter\def\csname PYG@tok@c\endcsname{\let\PYG@it=\textit\def\PYG@tc##1{\textcolor[rgb]{0.25,0.50,0.56}{##1}}}
\expandafter\def\csname PYG@tok@mf\endcsname{\def\PYG@tc##1{\textcolor[rgb]{0.13,0.50,0.31}{##1}}}
\expandafter\def\csname PYG@tok@err\endcsname{\def\PYG@bc##1{\setlength{\fboxsep}{0pt}\fcolorbox[rgb]{1.00,0.00,0.00}{1,1,1}{\strut ##1}}}
\expandafter\def\csname PYG@tok@mb\endcsname{\def\PYG@tc##1{\textcolor[rgb]{0.13,0.50,0.31}{##1}}}
\expandafter\def\csname PYG@tok@ss\endcsname{\def\PYG@tc##1{\textcolor[rgb]{0.32,0.47,0.09}{##1}}}
\expandafter\def\csname PYG@tok@sr\endcsname{\def\PYG@tc##1{\textcolor[rgb]{0.14,0.33,0.53}{##1}}}
\expandafter\def\csname PYG@tok@mo\endcsname{\def\PYG@tc##1{\textcolor[rgb]{0.13,0.50,0.31}{##1}}}
\expandafter\def\csname PYG@tok@kd\endcsname{\let\PYG@bf=\textbf\def\PYG@tc##1{\textcolor[rgb]{0.00,0.44,0.13}{##1}}}
\expandafter\def\csname PYG@tok@mi\endcsname{\def\PYG@tc##1{\textcolor[rgb]{0.13,0.50,0.31}{##1}}}
\expandafter\def\csname PYG@tok@kn\endcsname{\let\PYG@bf=\textbf\def\PYG@tc##1{\textcolor[rgb]{0.00,0.44,0.13}{##1}}}
\expandafter\def\csname PYG@tok@cpf\endcsname{\let\PYG@it=\textit\def\PYG@tc##1{\textcolor[rgb]{0.25,0.50,0.56}{##1}}}
\expandafter\def\csname PYG@tok@kr\endcsname{\let\PYG@bf=\textbf\def\PYG@tc##1{\textcolor[rgb]{0.00,0.44,0.13}{##1}}}
\expandafter\def\csname PYG@tok@s\endcsname{\def\PYG@tc##1{\textcolor[rgb]{0.25,0.44,0.63}{##1}}}
\expandafter\def\csname PYG@tok@kp\endcsname{\def\PYG@tc##1{\textcolor[rgb]{0.00,0.44,0.13}{##1}}}
\expandafter\def\csname PYG@tok@w\endcsname{\def\PYG@tc##1{\textcolor[rgb]{0.73,0.73,0.73}{##1}}}
\expandafter\def\csname PYG@tok@kt\endcsname{\def\PYG@tc##1{\textcolor[rgb]{0.56,0.13,0.00}{##1}}}
\expandafter\def\csname PYG@tok@sc\endcsname{\def\PYG@tc##1{\textcolor[rgb]{0.25,0.44,0.63}{##1}}}
\expandafter\def\csname PYG@tok@sb\endcsname{\def\PYG@tc##1{\textcolor[rgb]{0.25,0.44,0.63}{##1}}}
\expandafter\def\csname PYG@tok@k\endcsname{\let\PYG@bf=\textbf\def\PYG@tc##1{\textcolor[rgb]{0.00,0.44,0.13}{##1}}}
\expandafter\def\csname PYG@tok@se\endcsname{\let\PYG@bf=\textbf\def\PYG@tc##1{\textcolor[rgb]{0.25,0.44,0.63}{##1}}}
\expandafter\def\csname PYG@tok@sd\endcsname{\let\PYG@it=\textit\def\PYG@tc##1{\textcolor[rgb]{0.25,0.44,0.63}{##1}}}

\def\PYGZbs{\char`\\}
\def\PYGZus{\char`\_}
\def\PYGZob{\char`\{}
\def\PYGZcb{\char`\}}
\def\PYGZca{\char`\^}
\def\PYGZam{\char`\&}
\def\PYGZlt{\char`\<}
\def\PYGZgt{\char`\>}
\def\PYGZsh{\char`\#}
\def\PYGZpc{\char`\%}
\def\PYGZdl{\char`\$}
\def\PYGZhy{\char`\-}
\def\PYGZsq{\char`\'}
\def\PYGZdq{\char`\"}
\def\PYGZti{\char`\~}
% for compatibility with earlier versions
\def\PYGZat{@}
\def\PYGZlb{[}
\def\PYGZrb{]}
\makeatother

\renewcommand\PYGZsq{\textquotesingle}

\begin{document}

\maketitle
\tableofcontents
\phantomsection\label{index::doc}


Contents:


\chapter{Pyrosim Class Documentation}
\label{code:pyrosim-class-documentation}\label{code:module-pyrosim}\label{code::doc}\label{code:welcome-to-pyrosim-s-documentation}\index{pyrosim (module)}\index{PYROSIM (class in pyrosim)}

\begin{fulllineitems}
\phantomsection\label{code:pyrosim.PYROSIM}\pysiglinewithargsret{\sphinxstrong{class }\sphinxcode{pyrosim.}\sphinxbfcode{PYROSIM}}{\emph{playBlind=False, playPaused=False, evalTime=100, dt=0.05, xyz={[}0.8317, -0.9817, 0.8{]}, hpr={[}121, -27.5, 0.0{]}, debug=False}}{}
Python interface for ODE simulator
\paragraph{Attributes}

\noindent\begin{tabulary}{\linewidth}{|L|L|}
\hline

playBlind
&
(bool, optional) If True the simulation runs without graphics (headless) else if False the simulation runs with graphics (the default is False)
\\
\hline
playPaused
&
(bool, optional) If True the simulation starts paused else if False the simulation starts running. With simulation window in focus use Ctrl-p to toggle pausing the simulation. (the default is False)
\\
\hline
evalTime
&
(int, optional) The number of discrete steps in the simulation (the default is 100)
\\
\hline
dt
&
(float, optional) The time in seconds between physics world steps. Larger dt values create more unstable physics. (the default is 0.05)
\\
\hline
xyz
&
(list of 3 floats) The xyz position of the camera (default is {[}0.8317,-0.9817,0.8000{]})
\\
\hline
hpr
&
The heading, pitch, and roll of the camera (default is {[}121,-27.5,0.0{]})
\\
\hline
debug
&
(bool, optional) If True print out every string command sent through the pipe to the simulator (the default is False)
\\
\hline\end{tabulary}

\paragraph{Methods}
\index{Get\_Sensor\_Data() (pyrosim.PYROSIM method)}

\begin{fulllineitems}
\phantomsection\label{code:pyrosim.PYROSIM.Get_Sensor_Data}\pysiglinewithargsret{\sphinxbfcode{Get\_Sensor\_Data}}{\emph{sensorID=0}, \emph{svi=0}}{}
Get the post simulation data from a specified sensor
\begin{quote}\begin{description}
\item[{Parameters}] \leavevmode
\textbf{sensorID} : int , optional
\begin{quote}

the sensors ID tag
\end{quote}

\textbf{svi} : int , optional
\begin{quote}

The sensor value index. Certain sensors have multiple values 
(e.g. the position sensor) and the svi specifies which to 
access (e.g. for a position sensor, svi=0 corresponds to the
x value of that sensor)
\end{quote}

\item[{Returns}] \leavevmode
list of float
\begin{quote}

Returns the list of sensor values over the simulation.
\end{quote}

\end{description}\end{quote}

\end{fulllineitems}

\index{Send\_Bias\_Neuron() (pyrosim.PYROSIM method)}

\begin{fulllineitems}
\phantomsection\label{code:pyrosim.PYROSIM.Send_Bias_Neuron}\pysiglinewithargsret{\sphinxbfcode{Send\_Bias\_Neuron}}{\emph{neuronID=0}}{}
Send bias neuron to simulator
\begin{quote}\begin{description}
\item[{Parameters}] \leavevmode
\textbf{neuronID} : int, optional
\begin{quote}

User specified ID tag for the neuron
\end{quote}

\item[{Returns}] \leavevmode
bool
\begin{quote}

True if successful, False otherwise
\end{quote}

\end{description}\end{quote}

\end{fulllineitems}

\index{Send\_Box() (pyrosim.PYROSIM method)}

\begin{fulllineitems}
\phantomsection\label{code:pyrosim.PYROSIM.Send_Box}\pysiglinewithargsret{\sphinxbfcode{Send\_Box}}{\emph{objectID=0}, \emph{x=0}, \emph{y=0}, \emph{z=0}, \emph{length=0.1}, \emph{width=0.1}, \emph{height=0.1}, \emph{r=1}, \emph{g=1}, \emph{b=1}}{}
Send box body to the simulator
\begin{quote}\begin{description}
\item[{Parameters}] \leavevmode
\textbf{objectID} : int, optional
\begin{quote}

User specified body ID tag for the box
\end{quote}

\textbf{x} : float, optional
\begin{quote}

The x position coordinate of the center
\end{quote}

\textbf{y} : float, optional
\begin{quote}

The y position coordinate of the center
\end{quote}

\textbf{z} : float, optional
\begin{quote}

The z position coordinate of the center
\end{quote}

\textbf{length} : float, optional
\begin{quote}

The length of the box
\end{quote}

\textbf{width} : float, optional
\begin{quote}

The width of the box
\end{quote}

\textbf{height} : float, optional
\begin{quote}

The height of the box
\end{quote}

\textbf{r} : float, optional
\begin{quote}

The amount of the color red in the box (r in {[}0,1{]})
\end{quote}

\textbf{g} : float, optional
\begin{quote}

The amount of the color green in the box (g in {[}0,1{]})
\end{quote}

\textbf{b} : float, optional
\begin{quote}

The amount of the color blue in the box (b in {[}0,1{]})
\end{quote}

\item[{Returns}] \leavevmode
bool
\begin{quote}

True if successful, False otherwise
\end{quote}

\end{description}\end{quote}

\end{fulllineitems}

\index{Send\_Camera() (pyrosim.PYROSIM method)}

\begin{fulllineitems}
\phantomsection\label{code:pyrosim.PYROSIM.Send_Camera}\pysiglinewithargsret{\sphinxbfcode{Send\_Camera}}{\emph{xyz}, \emph{hpr}}{}
Sends camera position to simulator in eulerian coordinates
\begin{quote}\begin{description}
\item[{Parameters}] \leavevmode
\textbf{xyz} : list of floats
\begin{quote}

A length 3 list specifying the x,y,z position of the camera
in simulation
\end{quote}

\textbf{hpr} : list of floats
\begin{quote}

A length 3 list specifying the heading, pitch and roll of the camera
\end{quote}

\item[{Returns}] \leavevmode
None

\end{description}\end{quote}

\end{fulllineitems}

\index{Send\_Cylinder() (pyrosim.PYROSIM method)}

\begin{fulllineitems}
\phantomsection\label{code:pyrosim.PYROSIM.Send_Cylinder}\pysiglinewithargsret{\sphinxbfcode{Send\_Cylinder}}{\emph{objectID=0}, \emph{x=0}, \emph{y=0}, \emph{z=0}, \emph{r1=0}, \emph{r2=0}, \emph{r3=1}, \emph{length=1.0}, \emph{radius=0.1}, \emph{r=1}, \emph{g=1}, \emph{b=1}}{}
Send cylinder body to the simulator
\begin{quote}\begin{description}
\item[{Parameters}] \leavevmode
\textbf{objectID} : int, optional
\begin{quote}

User specified body ID tag for the cylinder (default is 0)
\end{quote}

\textbf{x} : float, optional
\begin{quote}

The x position coordinate of the center (default is 0)
\end{quote}

\textbf{y} : float, optional
\begin{quote}

The y position coordinate of the center (default is 0)
\end{quote}

\textbf{z} : float, optional
\begin{quote}

The z position coordinate of the center (default is 0)
\end{quote}

\textbf{r1} : float, optional
\begin{quote}

The orientation along the x axis. The vector {[}r1,r2,r3{]}
specify the direction of the long axis of the cylinder.
(default is 0)
\end{quote}

\textbf{r2} : float, optional
\begin{quote}

The orientation along the y axis. The vector {[}r1,r2,r3{]}
specify the direction of the long axis of the cylinder.
(default is 0)
\end{quote}

\textbf{r3} : float, optional
\begin{quote}

The orientation along the z axis. The vector {[}r1,r2,r3{]}
specify the direction of the long axis of the cylinder.
(default is 1)
\end{quote}

\textbf{length} : float, optional
\begin{quote}

The length of long axis of the cylinder (default is 1.0)
\end{quote}

\textbf{radius} : float, optional
\begin{quote}

The radius of the short axis of the cylinder (default is 0.1)
\end{quote}

\textbf{r} : float, optional
\begin{quote}

The amount of the color red in the box (r in {[}0,1{]})
\end{quote}

\textbf{g} : float, optional
\begin{quote}

The amount of the color green in the box (g in {[}0,1{]})
\end{quote}

\textbf{b} : float, optional
\begin{quote}

The amount of the color blue in the box (b in {[}0,1{]})
\end{quote}

\item[{Returns}] \leavevmode
bool
\begin{quote}

True if successful, False otherwise
\end{quote}

\end{description}\end{quote}

\end{fulllineitems}

\index{Send\_Developing\_Synapse() (pyrosim.PYROSIM method)}

\begin{fulllineitems}
\phantomsection\label{code:pyrosim.PYROSIM.Send_Developing_Synapse}\pysiglinewithargsret{\sphinxbfcode{Send\_Developing\_Synapse}}{\emph{sourceNeuronID=0}, \emph{targetNeuronID=0}, \emph{startWeight=0.0}, \emph{endWeight=0.0}, \emph{startTime=0.0}, \emph{endTime=1.0}}{}
Sends a synapse to the simulator

Developing synapses are synapses which change over time. 
The synapse will interpolate between the startWeight and endWeight
over the desired time range dictated by startTime and endTime.
startTime and endTime are in {[}0,1{]} where 0 maps to time step 0
and 1 maps to the evalTime of the simulation. Setting startTime
equal to endTime results in a discrete change from startWeight
to endWeight in the synapse at the specified time step. If
startTime \textgreater{}= endTime times are changed such that
endTime = startTime.
\begin{quote}\begin{description}
\item[{Parameters}] \leavevmode
\textbf{sourceNeuronID} : int, optional
\begin{quote}

The ID of the source neuron of the synapse
\end{quote}

\textbf{targetNeuronID} : int, optional
\begin{quote}

The ID of the target neuron of the synapse
\end{quote}

\textbf{startWeight} : float, optional
\begin{quote}

The starting edge weight of the synapse
\end{quote}

\textbf{endWeight} : float, optional
\begin{quote}

The ending edge weight of the synapse
\end{quote}

\textbf{startTime} : float, optional
\begin{quote}

The starting time of development. startTime in {[}0,1{]}
\end{quote}

\textbf{endTime} : float, optional
\begin{quote}

The ending time of development. endTime in {[}0,1{]}
\end{quote}

\textbf{Returns:}

\textbf{--------}

\textbf{bool}
\begin{quote}

True if successful, False otherwise
\end{quote}

\end{description}\end{quote}

\end{fulllineitems}

\index{Send\_Function\_Neuron() (pyrosim.PYROSIM method)}

\begin{fulllineitems}
\phantomsection\label{code:pyrosim.PYROSIM.Send_Function_Neuron}\pysiglinewithargsret{\sphinxbfcode{Send\_Function\_Neuron}}{\emph{neuronID=0}, \emph{function=\textless{}built-in function sin\textgreater{}}}{}
Send neuron to simulator which takes its value from the user defined function

The function is mapped to the specific time in the simulation based on both 
the discrete evaluation time and the dt space between time steps. For example
if evalTime=100 and dt=0.05 the function will be evaluated at {[}0,0.05,...,5{]}
\begin{quote}\begin{description}
\item[{Parameters}] \leavevmode
\textbf{neuronID} : int, optional
\begin{quote}

The user specified ID tag of the neuron
\end{quote}

\textbf{function} : function, optional
\begin{quote}

The function which defines the neuron value. Valid functions return
a single float value over the time domain.
\end{quote}

\item[{Returns}] \leavevmode
bool
\begin{quote}

True if successful, False otherwise
\end{quote}

\end{description}\end{quote}

\end{fulllineitems}

\index{Send\_Hidden\_Neuron() (pyrosim.PYROSIM method)}

\begin{fulllineitems}
\phantomsection\label{code:pyrosim.PYROSIM.Send_Hidden_Neuron}\pysiglinewithargsret{\sphinxbfcode{Send\_Hidden\_Neuron}}{\emph{neuronID=0}, \emph{tau=1.0}}{}
Send a hidden neuron to the simulator

Hidden neurons are basic neurons which can have inputs and outputs. 
They `hidden' between input neurons (sensors, bias, function) and 
output neurons (motors)
\begin{quote}\begin{description}
\item[{Parameters}] \leavevmode
\textbf{neuronID} : int, optional
\begin{quote}

The user specified ID tag of the neuron
\end{quote}

\textbf{tau} : float, optional
\begin{quote}

The `learning rate' of the neuron. Increasing tau increases
how much of value of the neuron at the current time step comes
from external inputs vs. the value of the neuron at the previous
time step
\end{quote}

\item[{Returns}] \leavevmode
bool
\begin{quote}

True if successful, False otherwise
\end{quote}

\end{description}\end{quote}

\end{fulllineitems}

\index{Send\_Joint() (pyrosim.PYROSIM method)}

\begin{fulllineitems}
\phantomsection\label{code:pyrosim.PYROSIM.Send_Joint}\pysiglinewithargsret{\sphinxbfcode{Send\_Joint}}{\emph{jointID=0}, \emph{firstObjectID=0}, \emph{secondObjectID=1}, \emph{x=0}, \emph{y=0}, \emph{z=0}, \emph{n1=0}, \emph{n2=0}, \emph{n3=1}, \emph{lo=-0.7853981633974483}, \emph{hi=0.7853981633974483}, \emph{speed=1.0}, \emph{torque=10.0}, \emph{positionControl=True}}{}
Send a hinge joint to the simulator
\begin{quote}\begin{description}
\item[{Parameters}] \leavevmode
\textbf{jointID} : int, optional
\begin{quote}

User specified  ID tag for the joint (default is 0)
\end{quote}

\textbf{firstObjectID} : int, optional
\begin{quote}

The body ID of the first body the joint is connected to.
If set equal to -1, the joint is connected to a point in
space (default is 0)
\end{quote}

\textbf{secondOjbectID} : int, optional
\begin{quote}

The body ID of the second body the joint is connected to.
If set equal to -1, the joint is connected to a point in
space (default is 1)
\end{quote}

\textbf{x} : float, optional
\begin{quote}

The x position coordinate of the joint (default is 0)
\end{quote}

\textbf{y} : float, optional
\begin{quote}

The y position coordinate of the joint (default is 0)
\end{quote}

\textbf{z} : float, optional
\begin{quote}

The z position coordinate of the joint (default is 0)
\end{quote}

\textbf{n1} : float, optional
\begin{quote}

The orientation along the x axis. The vector {[}n1,n2,n3{]}
specifies the axis about which the joint rotates
(default is 0)
\end{quote}

\textbf{n2} : float, optional
\begin{quote}

The orientation along the y axis. The vector {[}n1,n2,n3{]}
specifies the axis about which the joint rotates
(default is 0)
\end{quote}

\textbf{n3} : float, optional
\begin{quote}

The orientation along the z axis. The vector {[}n1,n2,n3{]}
specifies the axis about which the joint rotates
(default is 1)
\end{quote}

\textbf{lo} : float, optional
\begin{quote}

The lower limit in radians of the joint (default is -pi/4)
\end{quote}

\textbf{hi} : float, optional
\begin{quote}

The upper limit in radians of the joint (default is pi/4)
\end{quote}

\textbf{speed} : float, optional
\begin{quote}

The speed of the motor of the joint (default is 1.0)
\end{quote}

\textbf{positionControl} : bool, optional
\begin{quote}

True means use position control. This means the motor neuron
output is treated as a target angle for the joint to actuate
to. False means the motor neuron output is treated as a target
actuation rate.
\end{quote}

\item[{Returns}] \leavevmode
bool
\begin{quote}

True if successful, False otherwise
\end{quote}

\end{description}\end{quote}

\end{fulllineitems}

\index{Send\_Light\_Sensor() (pyrosim.PYROSIM method)}

\begin{fulllineitems}
\phantomsection\label{code:pyrosim.PYROSIM.Send_Light_Sensor}\pysiglinewithargsret{\sphinxbfcode{Send\_Light\_Sensor}}{\emph{sensorID=0}, \emph{objectID=0}}{}
Attaches a light sensor to a body in simulation
\begin{quote}\begin{description}
\item[{Parameters}] \leavevmode
\textbf{sensorID} : int, optional
\begin{quote}

The user defined ID of the sensor
\end{quote}

\textbf{objectID} : int, optional
\begin{quote}

The body ID of the body to connect the sensor to
\end{quote}

\item[{Returns}] \leavevmode
bool
\begin{quote}

True if successful
\end{quote}

\end{description}\end{quote}

\end{fulllineitems}

\index{Send\_Light\_Source() (pyrosim.PYROSIM method)}

\begin{fulllineitems}
\phantomsection\label{code:pyrosim.PYROSIM.Send_Light_Source}\pysiglinewithargsret{\sphinxbfcode{Send\_Light\_Source}}{\emph{objectIndex=0}}{}
Attaches light source to a body in simulation
\begin{quote}\begin{description}
\item[{Parameters}] \leavevmode
\textbf{objectIndex} : int, optional
\begin{quote}

The body ID of the body to attach the light to
\end{quote}

\item[{Returns}] \leavevmode
bool
\begin{quote}

True if successful, False otherwise
\end{quote}

\end{description}\end{quote}

\end{fulllineitems}

\index{Send\_Motor\_Neuron() (pyrosim.PYROSIM method)}

\begin{fulllineitems}
\phantomsection\label{code:pyrosim.PYROSIM.Send_Motor_Neuron}\pysiglinewithargsret{\sphinxbfcode{Send\_Motor\_Neuron}}{\emph{neuronID=0}, \emph{jointID=0}, \emph{tau=1.0}}{}
Send motor neurons to simulator

Motor neurons are neurons which connecto to a specified joint and 
determine how the joint moves every time step of simulation
\begin{quote}\begin{description}
\item[{Parameters}] \leavevmode
\textbf{neuronID} : int, optional
\begin{quote}

The user specified ID tag of the neuron
\end{quote}

\textbf{jointID} : int, optional
\begin{quote}

The joint ID tag of the joint we want the neuron to connect to
\end{quote}

\textbf{tau      :}
\begin{quote}

The `learning rate' of the neuron. Increasing tau increases
how much of value of the neuron at the current time step comes
from external inputs vs. the value of the neuron at the previous
time step
\end{quote}

\item[{Returns}] \leavevmode
bool
\begin{quote}

True if successful, False otherwise
\end{quote}

\end{description}\end{quote}

\end{fulllineitems}

\index{Send\_Position\_Sensor() (pyrosim.PYROSIM method)}

\begin{fulllineitems}
\phantomsection\label{code:pyrosim.PYROSIM.Send_Position_Sensor}\pysiglinewithargsret{\sphinxbfcode{Send\_Position\_Sensor}}{\emph{sensorID=0}, \emph{objectID=0}}{}
Attaches a position sensor to a body in simulation
\begin{quote}\begin{description}
\item[{Parameters}] \leavevmode
\textbf{sensorID} : int, optional
\begin{quote}

The user defined ID of the sensor
\end{quote}

\textbf{objectID} : int, optional
\begin{quote}

The body ID of the body to connect the sensor to
\end{quote}

\item[{Returns}] \leavevmode
bool
\begin{quote}

True if successful
\end{quote}

\end{description}\end{quote}

\end{fulllineitems}

\index{Send\_Proprioceptive\_Sensor() (pyrosim.PYROSIM method)}

\begin{fulllineitems}
\phantomsection\label{code:pyrosim.PYROSIM.Send_Proprioceptive_Sensor}\pysiglinewithargsret{\sphinxbfcode{Send\_Proprioceptive\_Sensor}}{\emph{sensorID=0}, \emph{jointID=0}}{}
Attaches a proprioceptive sensor to a joint in simulation

Proprioceptive sensors returns the angle of the joint at 
each time step
\begin{quote}\begin{description}
\item[{Parameters}] \leavevmode
\textbf{sensorID} : int, optional
\begin{quote}

The user defined ID of the sensor
\end{quote}

\textbf{jointID} : int, optional
\begin{quote}

The joint ID of the joint to connect the sensor to
\end{quote}

\item[{Returns}] \leavevmode
bool
\begin{quote}

True if successful
\end{quote}

\end{description}\end{quote}

\end{fulllineitems}

\index{Send\_Ray\_Sensor() (pyrosim.PYROSIM method)}

\begin{fulllineitems}
\phantomsection\label{code:pyrosim.PYROSIM.Send_Ray_Sensor}\pysiglinewithargsret{\sphinxbfcode{Send\_Ray\_Sensor}}{\emph{sensorID=0}, \emph{objectID=0}, \emph{x=0}, \emph{y=0}, \emph{z=0}, \emph{r1=0}, \emph{r2=0}, \emph{r3=1}}{}
Sends a ray sensor to the simulator connected to a body

Ray sensors return four values each time step, the distance and color (r,g,b).
\begin{quote}\begin{description}
\item[{Parameters}] \leavevmode
\textbf{sensorID} : int, optional
\begin{quote}

The user defined ID tag for the ray sensor
\end{quote}

\textbf{objectID} : int, optional
\begin{quote}

The body ID of the associated body the ray sensor is connected to. When this
body moves the ray sensor moves accordingly
\end{quote}

\textbf{x} : float, optional
\begin{quote}

The x position of the sensor
\end{quote}

\textbf{y} : float, optional
\begin{quote}

The y position of the sensor
\end{quote}

\textbf{z} : float, optional
\begin{quote}

The z position of the sensor
\end{quote}

\textbf{r1} : float, optional
\begin{quote}

The x direction of the sensor. The array {[}r1,r2,r3{]} is the direction the
ray sensor is pointing in the time step.
\end{quote}

\textbf{r2} : float, optional
\begin{quote}

The y direction of the sensor. The array {[}r1,r2,r3{]} is the direction the
ray sensor is pointing in the time step.
\end{quote}

\textbf{r3} : float, optional
\begin{quote}

The z direction of the sensor. The array {[}r1,r2,r3{]} is the direction the
ray sensor is pointing in the time step.
\end{quote}

\end{description}\end{quote}

\end{fulllineitems}

\index{Send\_Sensor\_Neuron() (pyrosim.PYROSIM method)}

\begin{fulllineitems}
\phantomsection\label{code:pyrosim.PYROSIM.Send_Sensor_Neuron}\pysiglinewithargsret{\sphinxbfcode{Send\_Sensor\_Neuron}}{\emph{neuronID=0}, \emph{sensorID=0}, \emph{sensorValueIndex=0}, \emph{tau=1.0}}{}
Sends a sensor neuron to the simulator

Sensor neurons are input neurons which take the value of their associated sensor
\begin{quote}\begin{description}
\item[{Parameters}] \leavevmode
\textbf{neuronID} : int, optional
\begin{quote}

The user defined ID of the neuron
\end{quote}

\textbf{sensorID} : int, optional
\begin{quote}

The associated sensor ID for the neuron
\end{quote}

\textbf{sensorValueIndex} : int, optional
\begin{quote}

The sensor value index is the offset index of the sensor. SVI is used for 
sensors which return a vector of values (position, ray sensors, etc.)
\end{quote}

\textbf{tau} : int, optional
\begin{quote}

not used for sensor neurons
\end{quote}

\item[{Returns}] \leavevmode
bool
\begin{quote}

True if successful, False otherwise
\end{quote}

\end{description}\end{quote}

\end{fulllineitems}

\index{Send\_Synapse() (pyrosim.PYROSIM method)}

\begin{fulllineitems}
\phantomsection\label{code:pyrosim.PYROSIM.Send_Synapse}\pysiglinewithargsret{\sphinxbfcode{Send\_Synapse}}{\emph{sourceNeuronID=0}, \emph{targetNeuronID=0}, \emph{weight=0.0}}{}
Sends a synapse to the simulator

Synapses are the edge connections between neurons
\begin{quote}\begin{description}
\item[{Parameters}] \leavevmode
\textbf{sourceNeuronID} : int, optional
\begin{quote}

The ID of the source neuron of the synapse
\end{quote}

\textbf{targetNeuronID} : int, optional
\begin{quote}

The ID of the target neuron of the synapse
\end{quote}

\textbf{weight} : float, optional
\begin{quote}

The edge weight of the synapse
\end{quote}

\end{description}\end{quote}

\end{fulllineitems}

\index{Send\_Touch\_Sensor() (pyrosim.PYROSIM method)}

\begin{fulllineitems}
\phantomsection\label{code:pyrosim.PYROSIM.Send_Touch_Sensor}\pysiglinewithargsret{\sphinxbfcode{Send\_Touch\_Sensor}}{\emph{sensorID=0}, \emph{objectID=0}}{}
Send touch sensor to a body in the simulator
\begin{quote}\begin{description}
\item[{Parameters}] \leavevmode
\textbf{sensorID} : int, optional
\begin{quote}

The user defined ID of the sensor
\end{quote}

\textbf{objectID} : int, optional
\begin{quote}

The body ID of the associated body
\end{quote}

\item[{Returns}] \leavevmode
bool
\begin{quote}

True if successful, False otherwise
\end{quote}

\end{description}\end{quote}

\end{fulllineitems}

\index{Send\_User\_Input\_Neuron() (pyrosim.PYROSIM method)}

\begin{fulllineitems}
\phantomsection\label{code:pyrosim.PYROSIM.Send_User_Input_Neuron}\pysiglinewithargsret{\sphinxbfcode{Send\_User\_Input\_Neuron}}{\emph{neuronID=0}, \emph{values=1}}{}
Send neuron to the simulator which takes user defined values at each time step
\begin{quote}\begin{description}
\item[{Parameters}] \leavevmode
\textbf{neuronID} : int, optional
\begin{quote}

The user specified ID tag of the neuron
\end{quote}

\textbf{values} : list of floats or float, optional
\begin{quote}

The user specified values for the neuron. If length of values \textless{} the number of
time steps, the values are continually looped through until every time step
has a corresponding value
\end{quote}

\item[{Returns}] \leavevmode
bool
\begin{quote}

True if successful, False otherwise
\end{quote}

\end{description}\end{quote}

\end{fulllineitems}

\index{Send\_Vestibular\_Sensor() (pyrosim.PYROSIM method)}

\begin{fulllineitems}
\phantomsection\label{code:pyrosim.PYROSIM.Send_Vestibular_Sensor}\pysiglinewithargsret{\sphinxbfcode{Send\_Vestibular\_Sensor}}{\emph{sensorID=0}, \emph{objectID=0}}{}
Connects a vestibular sensor to a body

Vestibular sensors return a bodies orrientation in space
\begin{quote}\begin{description}
\item[{Parameters}] \leavevmode
\textbf{sensorID} : int, optional
\begin{quote}

The user defined ID of the sensor
\end{quote}

\textbf{objectID} : int, optional
\begin{quote}

The body ID of the associated body
\end{quote}

\item[{Returns}] \leavevmode
bool
\begin{quote}

True if successful, False otherwise
\end{quote}

\end{description}\end{quote}

\end{fulllineitems}

\index{Start() (pyrosim.PYROSIM method)}

\begin{fulllineitems}
\phantomsection\label{code:pyrosim.PYROSIM.Start}\pysiglinewithargsret{\sphinxbfcode{Start}}{}{}
Starts the simulation

\end{fulllineitems}

\index{Wait\_To\_Finish() (pyrosim.PYROSIM method)}

\begin{fulllineitems}
\phantomsection\label{code:pyrosim.PYROSIM.Wait_To_Finish}\pysiglinewithargsret{\sphinxbfcode{Wait\_To\_Finish}}{}{}
Waits to for the simulation to finish and collects data
\begin{quote}\begin{description}
\item[{Returns}] \leavevmode
numpy matrix
\begin{quote}

A matrix of the sensor values for each time step of the simulation
\end{quote}

\end{description}\end{quote}

\end{fulllineitems}


\end{fulllineitems}



\chapter{Indices and tables}
\label{index:indices-and-tables}\begin{itemize}
\item {} 
\DUrole{xref,std,std-ref}{genindex}

\item {} 
\DUrole{xref,std,std-ref}{modindex}

\item {} 
\DUrole{xref,std,std-ref}{search}

\end{itemize}


\renewcommand{\indexname}{Python Module Index}
\begin{theindex}
\def\bigletter#1{{\Large\sffamily#1}\nopagebreak\vspace{1mm}}
\bigletter{p}
\item {\texttt{pyrosim}}, \pageref{code:module-pyrosim}
\end{theindex}

\renewcommand{\indexname}{Index}
\printindex
\end{document}
